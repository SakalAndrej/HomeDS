\section*{Declaration of Academic Honesty}
Hereby, I declare that I have composed the presented paper independently on my own and without any other resources than the ones indicated. All thoughts taken directly or indirectly from external sources are properly denoted as such.

This paper has neither been previously submitted to another authority nor has it been published yet. \\[1em]
Leonding, \duedateen \\[5em]
\ifthenelse{\isundefined{\firstauthor}}{}{\firstauthor}
\ifthenelse{\isundefined{\secondauthor}}{}{\kern-1ex, \secondauthor}
\ifthenelse{\isundefined{\thirdauthor}}{}{\kern-1ex, \thirdauthor}
\ifthenelse{\isundefined{\fourthauthor}}{}{\kern-1ex, \fourthauthor} \\[5em]

\begin{otherlanguage}{german}
\section*{Eidesstattliche Erkl�rung}
Hiermit erkl�re ich an Eides statt, dass ich die vorgelegte Diplomarbeit selbstst�ndig und ohne Benutzung anderer als der angegebenen Hilfsmittel angefertigt habe. Gedanken, die aus fremden Quellen direkt oder indirekt �bernommen wurden, sind als solche gekennzeichnet.

Die Arbeit wurde bisher in gleicher oder �hnlicher Weise keiner anderen Pr�fungsbeh�rde vorgelegt und auch noch nicht ver�ffentlicht. \\[1em]
Leonding, am \duedatede \\[5em]
\ifthenelse{\isundefined{\firstauthor}}{}{\firstauthor}
\ifthenelse{\isundefined{\secondauthor}}{}{\kern-1ex, \secondauthor}
\ifthenelse{\isundefined{\thirdauthor}}{}{\kern-1ex, \thirdauthor}
\ifthenelse{\isundefined{\fourthauthor}}{}{\kern-1ex, \fourthauthor} \\[5em]
\end{otherlanguage}

\begin{abstract}
Die HTL-Leonding besitzt schon einige Multimedia Systeme verstreut im ganzen Schulgeb�ude um Projekte, aktuelle News und �nderungen im Unterrichtsablauf anzuzeigen. Doch ein gro�er Schwachpunkt dieser Multimedia Systeme ist, dass der Prozess vom erstellen der Anzeige bis zum zuordnen welcher Bildschirm, welche Information anzeigen soll sehr kompliziert, und m�hselig ist. Sodass oftmals neue Informationen erst Versp�tet oder gar nicht angezeigt wird.

Unsere Diplomarbeit besch�ftigt sich mit dem erschaffen eines gemeinsames System zu entwickeln um einfach neue Supplierungen, Nachrichten oder Eilmeldungen auf allen Bildschirmen der HTL-Leonding anzuzeigen. Diese Systeme werden unter dem Begriff "Digital Signage System" zusammengefasst.

Concerning the content the following points shall be covered. 

\begin{enumerate}
	\item {\em Definition of the project:} What do we currently know about the topic or on which results can the work be based? What is the goal of the project? Who can use the results of the project?
	
	\item {\em Implementation:} What are the tools and methods used to implement the project?
	
	\item {\em Results:} What is the final result of the project?
\end{enumerate}
This list does not mean that the abstract must strictly follow this structure. Rather it should be understood in that way that these points shall be described such that the reader is animated  to dig further into the thesis.

Finally it is required to add a representative image which describes your project best. The image here shows Leslie Lamport the inventor of \LaTeX.



\end{abstract}

\begin{otherlanguage}{german}
\begin{abstract}
An dieser Stelle wird beschrieben, worum es in der Diplomarbeit geht. Die Zusammenfassung soll kurz und pr�gnant sein und den Umfang einer Seite nicht �bersteigen. Weiters ist zu beachten, dass hier keine Kapitel oder Abschnitte zur Strukturierung verwendet werden. Die Verwendung von Abs�tzen ist zul�ssig. Wenn notwendig, k�nnen auch Aufz�hlungslisten verwendet werden. Dabei ist aber zu beachten, dass auch in der Zusammenfassung vollst�ndige S�tze gefordert sind.

Bez�glich des Inhalts sollen folgende Punkte in der Zusammenfassung vorkommen: 

\begin{itemize}
	\item {\em Aufgabenstellung:} Von welchem Wissenstand kann man im Umfeld der Aufgabenstellung ausgehen? Was ist das Ziel des Projekts? Wer kann die Ergebnisse der Arbeit benutzen?
	
	\item {\em Umsetzung:} Welche fachtheoretischen oder -praktischen Methoden wurden bei der Umsetzung verwendet?
	
	\item {\em Ergebnisse:} Was ist das endg�ltige Ergebnis der Arbeit?
\end{itemize}
Diese Liste soll als Sammlung von inhaltlichen Punkten f�r die Zusammenfassung verstanden werden. Die konkrete Gliederung und Reihung der Punkte ist den Autoren �berlassen. Zu beachten ist, dass der/die LeserIn beim Lesen dieses Teils Lust bekommt, diese Arbeit weiter zu lesen.

Abschlie�end soll die Zusammenfassung noch ein Foto zeigen, das das beschriebene Projekt am besten repr�sentiert. Das folgende Bild zeigt Leslie Lamport, den Erfinder von \LaTeX.



\end{abstract}
\end{otherlanguage}

\section*{Acknowledgments}
If you feel like saying thanks to your grandma and/or other relatives.
