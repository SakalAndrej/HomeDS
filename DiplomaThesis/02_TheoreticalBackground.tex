\chapter{Theoretical Background}\label{cha:theoretical-background}
The details of the structure of your thesis have to be aligned with the supervising teacher. However, most of the theses require to have some description of the models used or some other theoretical background necessary to understand the rest of the text.

Since there is enough space here a table is added to show the basic usage of tables in a scientific document. Similarly to images these are also kept outside the normal text flow in a so-called floating body. Table~\ref{tab:types-of-floating-bodies} shows different options.

\begin{table}
	\begin{center}
	\begin{tabular}{|l|l|}
		\hline
		\cellcolor{Gray}\textcolor{White}{Body type} & \cellcolor{Gray}\textcolor{White}{Floats} \\
		\hline
		Image & Always \\
		\hline
		Table & Always \\
		\hline
		Algorithm & Sometimes \\
		\hline
	\end{tabular}
	\end{center}
	\caption{Different types of floating bodies}
	\label{tab:types-of-floating-bodies}
\end{table}