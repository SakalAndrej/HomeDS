\chapter{Einleitung}
\section{Ausgangssituation}
Die HTL-Leonding besitzt schon einige Multimedia Systeme verstreut im ganzen Schulgebäude um Projekte, aktuelle News und Änderungen im Unterrichtsablauf anzuzeigen. Doch ein großer Schwachpunkt dieser Multimedia Systeme ist, dass der Prozess vom Erstellen der Anzeige bis zum Zuordnen welcher Bildschirm welche Information anzeigen soll, sehr kompliziert und mühselig ist. So werden neue Informationen erst verspätet oder gar nicht angezeigt. 

\section{Ziele}
Ziel ist es, dass die Schulverwaltung möglichst schnell überall in der Schule Informationen, Warnungen oder Ankündigungen anzeigen kann. Die verschiedenen Multimediasysteme sollen einheitlich gesteuert und verwaltet werden können, um schnell alle Anzeigen beliebig zu verändern. So ist es auch ein Teilziel festzustellen, ob es möglich ist die derzeitig verwendeten Anzeigesysteme durch den XIBO Server zu ersetzen.

\section{Problemstellung}
Momentan wird um eine Anzeige zu ändern sehr viel Aufwand betrieben. Zum Beispiel wird eine neue Präsentation in Form von Folien oder Video zusammengeschnitten. Beispiel dafür ist die Anzeige im Eingangsbereich der Schule. Diese Vorgehensweise ist zeitaufwendig und werden Änderungen vorgenommen, kann  die alte Präsentation oder das Video meistens verworfen werden.

