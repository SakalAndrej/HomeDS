\chapter{Einleitung}
\section{Ausgangssituation}
Die HTL-Leonding besitzt schon einige Multimedia Systeme verstreut im ganzen Schulgeb�ude um Projekte, aktuelle News und �nderungen im Unterrichtsablauf anzuzeigen. Doch ein gro�er Schwachpunkt dieser Multimedia Systeme ist, dass der Prozess vom erstellen der Anzeige bis zum zuordnen welcher Bildschirm, welche Information anzeigen soll sehr kompliziert, und m�hselig ist. Sodass oftmals neue Informationen erst Versp�tet oder gar nicht angezeigt wird. 

\section{Ziele}
Ziel ist es, dass es der Schulverwaltung m�glich ist Informationen, Warnungen oder Ank�ndigungen m�glichst schnell �berall in der Schule anzuzeigen. Die verschiedenen Multimediasysteme sollen einheitlich gesteuert und verwaltet werden k�nnen um schnell alle Anzeigen beliebig zu ver�ndern. So ist es auch ein Teilziel festzustellen ob es m�glich die derzeitig verwendeten Anzeigesysteme durch den XIBO Server zu ersetzen.

\section{Problemstellung}
Momentan wird um eine Anzeige zu �ndern sehr viel Aufwand betrieben, zum Beispiel wird eine neue Pr�sentation erstellt in Form von Folien oder ein Video zusammengeschnitten Beispiel daf�r ist die Anzeige im Eingangsbereich der Schule. Diese Vorgehensweise ist zeitaufwendig und werden �nderungen vorgenommen, kann man die alte Pr�sentation oder das Video meistens verwerfen.

