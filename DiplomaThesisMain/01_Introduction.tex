\chapter{Einleitung}
\section{Ausgangssituation}
Die HTL-Leonding besitzt schon einige Multimedia Systeme verstreut im ganzen Schulgeb�ude um Projekte, aktuelle News und �nderungen im Unterrichtsablauf anzuzeigen. Doch ein gro�er Schwachpunkt dieser Multimedia Systeme ist, dass der Prozess vom Erstellen der Anzeige bis zum Zuordnen welcher Bildschirm welche Information anzeigen soll, sehr kompliziert und m�hselig ist. So werden neue Informationen erst versp�tet oder gar nicht angezeigt. 

\section{Ziele}
Ziel ist es, dass die Schulverwaltung möglichst schnell �berall in der Schule Informationen, Warnungen oder Ank�ndigungen anzeigen kann. Die verschiedenen Multimediasysteme sollen einheitlich gesteuert und verwaltet werden k�nnen, um schnell alle Anzeigen beliebig zu ver�ndern. So ist es auch ein Teilziel festzustellen, ob es m�glich ist die derzeitig verwendeten Anzeigesysteme durch den XIBO Server zu ersetzen.

\section{Problemstellung}
Momentan wird um eine Anzeige zu �ndern sehr viel Aufwand betrieben. Zum Beispiel wird eine neue Pr�sentation in Form von Folien oder Video zusammengeschnitten. Beispiel daf�r ist die Anzeige im Eingangsbereich der Schule. Diese Vorgehensweise ist zeitaufwendig und werden �nderungen vorgenommen, kann  die alte Pr�sentation oder das Video meistens verworfen werden.

