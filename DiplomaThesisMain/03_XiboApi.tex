\chapter{XIBO-Server}
\section{Beschreibung}
Als zentrale Steuereinheit wird ein XIBO-Server verwendet. Der XIBO-Server bietet die benötigten Funktionalitäten wie zum Beispiel:
\begin{itemize}
	\item {\em Medieninhalte abspielen} 
	\item {\em Zeitsteuerung der anzuzeigenden Informationen}  
	\item {\em Verteilung der Anzeigedaten an die verbundenen Clients} 
\end{itemize}
Es gibt zwei Möglichkeiten den Funktionsumfang des XIBO-Servers zu nutzen. Zum einen über das Server interne Web-Interface, die andere Möglichkeit ist es den Server über die eingebaute REST-Schnittstelle anzusprechen.
\cite{xibo-server}
\section{API}
Die API des XIBO-Servers ist mittels Swagger dokumentiert. Diese Dokumentation deckt die Grundfunktionalitäten und die Form der Anfragen ab. Die Schnittstelle des Servers dient als wesentliches Verbindungsstück zwischen der eigens entwickelten Steuerungssoftware und dem XIBO-Server. Wie der XIBO-Server die verschiedenen Anfragen verarbeitet und entgegen nimmt wurde, bevor die Implementierung des Java-EE-Servers begonnen wurde, mittels Postman getestet. Diese Vorgehensweise war Nötig um festzustellen ob das XIBO-System über REST-API ausreichend konfigurierbar und im operativen Betrieb steuerbar ist. Die Anfragen an den Server wurden im Java Code durch die ''libary'' OkHttp3 übernommen.

\cite{swagger}
\cite{postman}
\cite{Okhttp3}
\section{Authentifizierung}

Die Authentifizierung einer Client-Applikation per REST-Anfrage am XIBO-Server erfolgt über OAuth2 , also mittels Access Token.

Zunächst ist am XIBO-Server ein ''Application''-Objekt im Webinterface zu erstellen. Beim erstellen des ''Application''-Objekts können die Berechtigungen für den Client festgelegt werden. Nachdem das Objekt erstellt wurde stellt dieses ein ''Client Secret'' zur Verfügung. Dieses ist für jeden Client eindeutig. 

Damit ein externer Client auf den XIBO-Server per REST-Request zugreifen beziehungsweise Anweisungen an diesen geben kann, wird ein POST-Request mit folgenden Parametern abgesetzt.

Die Parameter: 
\begin{itemize}
	\item {\em Client\_ID:} XIBO-Server Anwendungen neue Anwendung
	\item {\em Client\_Secret:}  XIBO-Server Anwendungen neue Anwendung
	\item{\em grant\_type:} Muss in der Form ''&grant\_type=client\_credentials''
\end{itemize}

Die über den POST-Request erhaltene Antwort liefert einen Access Token welcher 60 Minuten gültig ist und nach Ablauf erneuert werden muss um weiter über die API zu kommunizieren.
Dieser Token muss bei jedem Request an den XIBO-Server im Header des Requests übergeben werden damit der XIBO-Server feststellen kann ob es sich um einen registrierten Client handelt.

HomeDS\HomeDsBackend\src\main\java\at\htl\utils\AuthentificationHandler.java

Um den die Authentifizierung zu automatisieren wurde eine Java Klasse entwickelt. Funktionsweise dieser wird nachfolgend geschildert: 

Um eine Verbindung zum XIBO-Server herstellen zu können wird eina URL und eine HttpURLConnection deklariert. Im Anschluss wird die URL mit der richtigen Adresse belegt. Die httpURLConnection wird über den Befehl ''httpURLConnection.openConnection()'' dazu angewiesen eine Verbindung aufzubauen. Durch die Anweisung ''httpURLConnection.setDoOutput(true)'' wird der Connection mitgeteilt das als Antwort Daten erhalten werden. Die Art der Anfrage wird als POST-Request festgelegt. Um die für die Authentifizierung am XIBO-Server geforderten Parameter ''client\_id'', ''client\_secret'' und ''grant\_type'' übergeben zu können wird ein ''DataOutputStream'' deklariert und mit dem Benötigten werten versehen. Anschließend werden über den Befehl  ''DataoutputStrem.flush'' wird dem ''DataOutputStream'' mitgeteilt, dass er die Daten über die Verbindung senden soll.Um auftretende Fehler besser finden zu können werden der Übergeben Request-Body, die URL über die der Request durchgeführt wurde und der erhaltene Response-Code im Log-Fenster ausgegeben. Die Daten die anschließend vom XIBO-Server als Antwort erhalten werden, werden durch einen ''BufferedReader'', dieser bekommt bei der Instanziierung den ''InputStream'' der ''httpURLConnection'' übergeben, entgegengenommen. Solange vom Server Daten erhalten werden wird über einen ''StringBuilder'' ein String um jene erhaltenen Daten erweitert. Im Anschluss wird die ''BufferedReader'' Verbindung geschlossen und der erhaltene ''Access\_Token'' wird im Log-Fenster angezeigt. Als nächstes wird der erhaltene Token per ''return'' Statement als Ergebniss der Methode übergeben. Abschließend wird im ''finally'' Block überprüft ob die ''HttpURLConnection'' noch geöffnet beziehungsweise vorhanden ist, sollte dies der Fall sein so wird die Verbindung geschlossen.






