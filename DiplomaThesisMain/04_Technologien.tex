\chapter{Verwendente Technologien}
\section{Git und GitHub}
Um dynamisch als Team arbeiten zu können, verwenden wir Software zur Versionsverwaltung. Hierbei handelt es sich um Git. 
Github ist die verwendete Online-Plattform, auf der Benutzer ihre Projekte gratis als Repository speichern. Dies ermöglicht einfaches arbeiten im Team und verhindert in den meisten Fällen Zusammenführungskonflikte. Mittels Git lässt sich auch leicht zurückverfolgen welches, Teammitglied welche Änderungen gemacht hat und im Notfall ist es auch möglich diese Änderungen wieder rückgängig zu machen.

Verwendet wird GitHub für die gesamte Diplomarbeit, sowohl für die Versionierung der Dokumente, als auch  die einzelnen Applicationen. Um sicherzustellen, das keine Konflikte durch paralleles arbeiten entstehen, wird in Branches gearbeitet. Diese Branches wurden erstellt, wenn ein neues Arbeitspaket begonnen wurde, zum Beispiel die Android-App.

Bild Github und verweis Git/GitHub

\section{Android}

Android ist ein Betriebssystem für mobile Endgeräte, spezialisiert für Touch-Anwendungen. Ziel ist es das Endgerät möglichst intuitiv und flexibel bedienen zu können. Mit Android ist es möglich open-source Applicationen zu erstellen die ein großes Publikum erreichen. Google stellt auch einen Markt zur Verfügung in dem die Applicationen gratis oder auch gegen Entgelt erworben werden können.
Diese Aspekte: open-source, gratis und großes Publikum, waren ausschlaggebend dafür, dass die Applicationen in Android implementiert wurden. Als Programmiersprache wurde JAVA verwendet.




\section{Java Enterprise Edition}\label{sec:javaee}
Java Platform, Enterprise Edition oder abgekürzt auch Java EE ist die technische nähere Beschreibung einer Softwarearchitektur, die programmierte Java Anwendungen ausführt.
(weiter ausführen)

QUELLE: https://de.Wikipedia.org/wiki/Java_Platform,_Enterprise_Edition

\section{JSF - Java Server Faces}\label{sec:javaee}