\chapter{Android Application}
\section{Einleitung}
Um eine höchst mögliche Reichweite an Endgeräten zu erzielen, wurde eine Android Applikation entwickelt, mit der die wichtigsten Funktionen abgedeckt werden. Zum Beispiel das Wechseln der DataSets des Digital Signage Servers. Wichtig war es die Applikation möglichst einfach und leicht bedienbar zu gestalten, um auch Erstbenutzern die Bedienung zu erleichtern. 
Prototyp für die aktuelle Applikation war eine Anwendung für Android, die direkt mit dem Digital Signage Server kommuniziert, da sehr schnell klar wurde, dass das Steuern des Servers direkt über eine Applikation auf Android Basis viel zu umständlich ist. Es wurde eine weitere Mobile Applikation entwickelt, welche über das ''HomeDsBackend'' kommuniziert, um das Funktionsspektrum zu erweitern und die Applikation möglichst kompakt zu gestalten. Beispielsweise ist es durch diese Aufteilung nicht mehr nötig eine Datenbank in der Applikation zu haben. Somit erleichtert es auch Applikationen für andere Betriebssysteme zu implementieren. 


\section{Anforderungen}
Die Anforderungen an die Android Applikation weichen von den Anforderungen an das ''Backend'' leicht ab, da das ''Backend'' die gesamte Zeitsteuerung- und Datenbankfunktionalität übernimmt. Es besteht die Möglichkeit, den Server über die Website des ''Backend'' zu steuern, beziehungsweise über die Applikation auf Android Basis. 

\begin{itemize}
	\item {\em Eilmeldungen:} Dem Benutzer soll es möglich sein Nachrichten in einem Ticker auf den Bildschirmen anzuzeigen.
	
	\item {\em Medien Wiedergabe:} Medien die Lokal am Digital Signage Server liegen sollen abgespielt werden können.
		
	\item {\em Authentifizierung automatisieren (Prototyp Applikation):} Die Authentifizierung am Digital Signage Server soll automatisiert werden, um nicht vom Benutzer durchgeführt werden zu müssen.  
		
\end{itemize}

\section{Verwendete Technologien}
Android wird mit der API(Englisch: application programming interface / Deutsch: Anwendungsprogrammierschnittstelle), in der Version 26 (Oreo / Android 8.0) verwendet. 
VERWEIS ANDROID !!!!
Als Erweiterung für die Http Anfragen an den Digital Signage Server und das HomeDsBackend wird OkHttp3 verwendet.

\section{Struktur}
\begin{itemize}
	\item {\em activity:} Alle Activities die für die Anwendung benötigt werden. Als Beispiel die MainActivity
	
	\item {\em adapter:} Hier befinden sich alle Adapter für die RecyclerViews.
	
	\item {\em apiClient:} Beinhaltet die Klasse ''RequestHelper'' mit der die Anfragen an den Digital Signage Server beziehungsweise an das HomeDsBackend vereinfacht werden.
	
	\item {\em entity:} Jene Klassen die als Models für die Anwendung benötigt werden. 
	STÜTZ!!!
	
	\item {\em enumeration:} Enumerationen welche die Android Basierte Applikation verwendet. Zum Beispiel das ''RequestTypeEnum''.
	
	\item {\em fragment:} Alle Fragments die für das User Interface benötigt werden. Beispiel hierfür ist das Fragment ''NewsOverviewFragment'', welches alle DataSets die am Server sind anzeigt.
	
	\item {\em viewholder:} Beinhaltet alle ''ViewHolder'' die für die Verschiedenen ''RecyclerViews'' benötigt werden. 
	  
		
\end{itemize}

\section{Main-Activity}
Die ''MainActivitiy'' ist die Einzige ''Activity'' die in der Android Applikation benötigt wird. Hauptaufgabe der ''MainActivity'' ist es, zwischen den einzelnen ''Fragments'' zu navigieren. Enthalten sind dazu jene Methoden, welche mit dem ''SupportFragmentManager'' die einzelnen Fragments im ''container\_main'' austauschen. Der ''container\_main'' ist ein ''ConstraintLayout'' welches in der zur ''MainActivity'' gehörenden Layout Ressource ''activity\_main.xml'' mit der Identifikationsnummer ''container\_main'' belegt wurde. Zudem wird die Klasse durch das Interface ''AppCompatActivity'' erweitert und implementiert die ''OnFragmentInteractionListener'' der Fragmente die in der Applikation verwendet werden.


\title{onCreate}
Hier wird die ''Main\_Activity'' als ''ContentView'' gesetzt, zudem wird mittels ''SupportFragmentManager'' das ''HomeScreenFragment'' dem ''container\_main'' zugewiesen und angezeigt. Die statische Variable ''instance'', vom Datentyp ''MainActivity'', wird auf die aktuelle Instanz der Klasse zugewiesen. 
	
\title{getInstance}
 Ist der ''Getter'' für die statische Variable ''instance'' und übergibt die aktuelle Instanz der Klasse.
	
\title{''Fragment'' Austausch Methoden}
 Sind jene Methoden die für das Austauschen der einzelnen ''Fragments'' im ''container\_main'' zuständig sind. Dies geschieht mittels ''SupportFragmentManager'' welcher immer die ''Fragments'' im ''container\_main'' anzeigt und dem ''BackStack'', welcher für die Rückwerts Navigation (mittels retour Knopf des Mobilen Endgerätes) in der Applikation zuständig ist,  hinzufügt. Manche Methoden übergeben zudem noch ein ''Bundle'' an das erstellte ''Fragment'', welches Objekte enthält die in nächsten ''Fragment'' benötigt werden. Erkennungsmerkmal dieser Methoden ist das englische Verb ''open'' am beginn des Methodennamens. Als Namensbeispiel hierfür wird die Methode ''openNewsEditFragment'' herangezogen.





\section{DataSet verwaltung}
Die beiden Fragmente  ''NewsOverviewFragment'' und ''NewsEditFragment'' sind für die Verwaltung der ''DataSets'' zuständig. Im Fragment ''NewsOverviewFragment'' wird eine Übersicht über alle vorhandenen ''DataSets'' gegeben. Das ''NewsEditFragment'' Fragment wird verwendet um vorhandene ''DataSets'' zu bearbeiten oder neue ''DataSets'' erstellen zu können. 


\title{NewsOverviewFragment}
Die Layout Ressource des Fragment enthält eine ''RecyclerView'' und einen ''FloatingActionButton''. Die Logik der Anzeige ist in der Java Klasse ''NewsOverviewFragment'' implementiert. Die meisten Methoden dieser Klasse werden generisch beim Erstellen eines Fragments erzeugt. Die einzige Methode die überschrieben wurde ist die Methode ''onCreateView''. Beim ausführen dieser Methode wird zuerst eine ''View'' erstellt welche das ''fragment_news_overview'' Layout zugewiesen bekommt. Anschließend wird eine ''RecyclerView'' und ein ''FlaootingActionButton'' erstellt und den zugehörigen Anzeige Elementen zugewiesen. Der ''FlaootingActionButton'' erhält einen ''OnClickListener'' über diesen wird in der ''MainActivity'' eine Methode aufgerufen die ein neues ''NewsEditFragment'' anzeigt, um ein neues ''DataSet'' zu erstellen. 
