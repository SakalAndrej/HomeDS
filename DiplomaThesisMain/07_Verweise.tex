\chapter*{Arbeitsaufteilung}
\subsubsection{Arbeit von Sakal Andrej}
\begin{normalsize}
Sakal Andrej's Arbeit im Zuge der Diplomarbeit bestand darin, den Teil des JavaEE Servers zu übernehmen. Er kümmerte sich um die gesamte serverseitge Anwendung, diese enthält einen REST-Service, die Kommunikation mit der MySQL-Datenbank und eine Java Server Faces Weboberfläche. Zusätzlich kümmerte er sich auch um die continuous integration, die mithilfe von Jenkins realisiert wurde.
\end{normalsize}

\subsubsection{Arbeit von Hofmann Felix}
\begin{normalsize}
Hofmann Felix's Teil im Zuge der Diplomarbeit bestand darin den gesamten Teil der Android Applikation zu übernehmen. Er kümmerte sich um die REST-Zugriffe, die Authentifizierung und das Designe von der Android Applikation.
\end{normalsize}

\subsubsection{Genaue Aufteilung der Arbeit}
\begin{normalsize}
Die folgende Tabelle zeigt eine genaue Übersicht der Arbeitsaufteilung der schriftlichen Arbeit. Hier werden die Kapitel den Arbeiter genau zugeteilt. Dabei werden die Bearbeiter mithilfe ihrer Initialen abgekürzt (SA = Sakal Andrej, HF = Hofmann Felix).
\end{normalsize}

\vspace{0.5cm}
\begin{longtable}{| l | l | c |}
\multicolumn{2}{l}{\textbf{Kapitel}} & \multicolumn{1}{c}{\textbf{Bearbeiter}} \\ \hline
\multicolumn{3}{|l|}{\textbf{1 Einleitung}} \\
\multicolumn{2}{|l|}{1.1 Ausgangssituation} & HF \\
\multicolumn{2}{|l|}{1.2 Problemstellung} & SA \\
\multicolumn{2}{|l|}{1.3 Aufgabenstellung} & HF \\
\multicolumn{2}{|l|}{1.4 Ziele} & SA \\ \hline

\multicolumn{3}{|l|}{\textbf{2 Digital Signage \& XIBO}} \\ \hline
\multicolumn{2}{|l|}{2.1 Was ist Digital Sigange?} & SA \\
\multicolumn{2}{|l|}{2.2 Digital Signage Anwendungen} & SA \\
\multicolumn{2}{|l|}{2.3 Was ist XIBO?} & SA \\
\multicolumn{2}{|l|}{2.4 Weboberfläche des XIBO} & SA \\
\multicolumn{2}{|l|}{2.5 Designen mit XIBO} & SA \\ \hline

\multicolumn{3}{|l|}{\textbf{3 XIBO-Server}} \\ \hline
\multicolumn{2}{|l|}{3.1 Beschreibung} & HF
\multicolumn{2}{|l|}{3.3 API} & HF \\ \hline
\multicolumn{2}{|l|}{3.3 Authentifizierung} & HF \\ \hline

\multicolumn{3}{|l|}{\textbf{4 Verwendete Technologien}} \\ \hline
\multicolumn{2}{|l|}{4.1 Git und GitHub} & HF \\
\multicolumn{2}{|l|}{4.2 Android} & HF \\
\multicolumn{2}{|l|}{4.3 Java Enterprise Edition} & SA \\
\multicolumn{2}{|l|}{4.4 JSF - Java Server Faces} & HF \\
\multicolumn{2}{|l|}{4.5 IntelliJ IDEA} & HF \\
\multicolumn{2}{|l|}{4.6 Android Studio} & HF \\
\multicolumn{2}{|l|}{4.7 IDraw IO} & SA \\ \hline

\multicolumn{3}{|l|}{\textbf{5 HomeDS - Server}} \\ \hline
\multicolumn{2}{|l|}{5.1 Einleitung} & SA \\
\multicolumn{2}{|l|}{5.2 Anforderungen an den HomeDS Server} & SA \\
\multicolumn{2}{|l|}{5.3 Komponenten des HomeDS Server} & SA \\
\multicolumn{2}{|l|}{5.4 Funktionen des JavaEE} & SA \\
\multicolumn{2}{|l|}{5.5 Funktionen des JavaEE - Technischer Hintergrund} & SA \\ \hline

\multicolumn{3}{|l|}{\textbf{6 Android Applikation - Server}} \\ \hline
\multicolumn{2}{|l|}{6.1 Einleitung} & HF \\
\multicolumn{2}{|l|}{6.2 Anforderungen} & HF \\
\multicolumn{2}{|l|}{6.3 Struktur} & HF \\
\multicolumn{2}{|l|}{6.4 Benutzerhandbuch} & HF \\
\multicolumn{2}{|l|}{6.5 MainBottomNavigationActivity und OverviewFragment} & HF \\
\multicolumn{2}{|l|}{6.6 DataSet Verwaltung} & HF \\
\multicolumn{2}{|l|}{6.7 Mediaplayer} & HF \\
\multicolumn{2}{|l|}{6.8 Strukturplan} & HF \\
\multicolumn{2}{|l|}{6.9 Request-Helper} & HF \\ \hline

\multicolumn{3}{|l|}{\textbf{7 Continuous Integration}} \\ \hline
\multicolumn{2}{|l|}{6.1 Einleitung} & SA \\
\multicolumn{2}{|l|}{6.2 Was ist CI?} & SA \\
\multicolumn{2}{|l|}{6.3 Wieso CI?} & SA \\
\multicolumn{2}{|l|}{6.4 Wie kann CI realisiert werden?} & SA \\
\multicolumn{2}{|l|}{6.5 Jenkins installieren} & SA \\
\multicolumn{2}{|l|}{6.6 Jenkins CI konfigurieren} & SA \\
\end{longtable}

\begin{thebibliography}{9}

\bibitem{xibo-server} 
Xibo Open Source Digital Signage (abgerufen am 26.03.2018)
\newblock URL: {\small \url{https://xibo.org.uk/}}

\bibitem{swagger} 
Swagger Dokumentation (abgerufen am 09.03.2018)
\newblock URL: {\small \url{https://swagger.io/docs/}}

\bibitem{postman} 
Postman Dokumentation (abgerufen am 12.02.2018)
\newblock URL: {\small \url{https://www.getpostman.com/docs/v6/www.getpostman.com/docs/v6/}}

\bibitem{oAuth2} 
OAuth2 Official Website (abgerufen am 15.03.2018)
\newblock URL: {\small \url{https://oauth.net/2/}}

\bibitem{OkHttp3} 
okHttp3 Dokumentation (abgerufen am 17.01.2018)
\newblock URL: {\small \url{https://square.github.io/okhttp/3.x/okhttp/}}

\bibitem{httpurlconnection} 
HttpUrlConnection Dokumentation (abgerufen am 15.03.2018)
\newblock URL: {\small \url{https://developer.android.com/reference/java/net/HttpURLConnection.html}}

\bibitem{differenceeese} 
Unterschied JavaEE und JavaSE (abgerufen am 28.03.2018)
\newblock URL: {\small \url{https://docs.oracle.com/javaee/6/firstcup/doc/gkhoy.html}}

\bibitem{wikijavaee} 
JavaEE (abgerufen am 21.03.2018)
\newblock URL: {\small \url{https://de.Wikipedia.org/wiki/Java_Platform,_Enterprise_Edition}}

\bibitem{wikijsf} 
Java Server Faces (abgerufen am 28.03.2018)
\newblock URL: {\small \url{https://de.wikipedia.org/wiki/JavaServer_Faces}}

\bibitem{wikiintelij} 
IntelliJ IDEA (abgerufen am 27.03.2018)
\newblock URL: {\small \url{https://de.wikipedia.org/wiki/IntelliJ_IDEA}}

\bibitem{androidstudio} 
Android Studio (abgerufen am 28.03.2018)
\newblock URL: {\small \url{https://developer.android.com/studio/features.html}}

\bibitem{drawio} 
draw.io - Grafik Online-Tool (abgerufen am 01.03.2018)
\newblock URL: {\small \url{https://www.draw.io/}}

\bibitem{Java-Lamda-Expressions} 
Java-EE-Lamda Expressions Oracle start Page.
\newblock URL: {\small \url{http://www.oracle.com/webfolder/technetwork/tutorials/obe/java/Lambda-QuickStart/index.html}}
\end{thebibliography}


